\documentclass[11pt,a4paper]{article}

\usepackage[utf8]{inputenc}
\usepackage{ngerman}
\usepackage{amsmath}
\usepackage[usenames,dvipsnames,svgnames,table]{xcolor}
% usenames allows you to use names of the default colors, the same 16 base colors as used in HTML. The dvipsnames allows you access to more colors, another 64, and svgnames allows access to about 150 colors. The initialization of "table" allows colors to be added to tables by placing the color command just before the table. If you need more colors, then you may also want to look at adding the x11names to the initialization section as well, this offers more than 300 colors, but you need to make sure your xcolor package is the most recent you can download.

\usepackage{graphicx}
\graphicspath{{bilder/}}
\usepackage[autostyle=true,german=quotes]{csquotes}
\usepackage{enumitem}

\renewcommand{\familydefault}{\sfdefault}

\date{\today}
\author{Marvin Janosch}
\title{Sitzungsprotokolle}


\begin{document}
	\maketitle
	\pagebreak
	\tableofcontents
	
	%\begin{abstract}
	%	\begin{center}
	%		Zusammenfassung!
	%	\end{center}
	%\end{abstract}
	
	\pagebreak
\section[Kundeninterview - 2017-10-18]{Kundeninterview - Beschreibung der Anforderungen}

\begin{enumerate}
	\item Datum: 2017-10-18
	\begin{enumerate}[label*=\arabic*.]
		\item Beginn: 15:00 Uhr
		\item Ende: 15:44 Uhr
	\end{enumerate}

	\item Teilnehmer:
	\subitem Jan~Amann, Marvin~Janosch, Dennis~Szczepanski, Jan~Philip~Wahle

	\item Protokollant: Jan~Philip~Wahle
\end{enumerate}
\ \\
\textbf{Vorbesprechung zur Projektplanung und Informationssammlung zur Erstellung des Lastenhefts}

\begin{itemize}
	\item Codeverwaltungssystem einheitlich auswählen: git, github. (Khalid)
	\item Synchronisation aller Sitzungs- sowie Zeitprotokolle.
	\item Frontend soll folgendes beinhalten:
	\begin{itemize}
		\item Map Editor:
		\begin{itemize}
			\item Graphische Oberfläche zum Erstellen von Strecken
			\item Speicher- und Ladefunktion
			\item Statische Objekte und Hindernisse
			\item Dynamische Objekte und Hindernisse
		\end{itemize}
		\item 2D Simulation
		\item 3D Visualisierung
	\end{itemize}
	\item Backend soll folgendes beinhalten:
	\begin{itemize}
		\item Sensorik des Fahrzeugs
		\item Physikalische Trägheit
		\item Selbstlernendes Netzwerk (Tensorflow)
	\end{itemize}
\end{itemize}
	\pagebreak
\section[Gruppentreffen - 2017-10-24]{Gruppentreffen}

\begin{enumerate}
	\item Datum: 2017-10-24
	\begin{enumerate}[label*=\arabic*.]
		\item Beginn: 14:00 Uhr
		\item Ende: 14:45 Uhr
	\end{enumerate}
	
	\item Teilnehmer:
	\subitem Jan~Amann, Khalid~Bellouch, Marvin~Janosch, Dennis~Szczepanski, Jan~Philip~Wahle
	
	\item Protokollant: Dennis~Szczepanski
\end{enumerate}
\ \\

\begin{itemize}
	\item Anlegen einer Codefibel bis zum 01.11.2017
	\item Überarbeiten des Pflichtenhefts und Glossars bis zum 01.11.2017
	\item Diskussion über Grundgerüst der Strecke
	\begin{itemize}
		\item Strecke als mehrdimensionales Grid
		\item Statische Objekte $\rightarrow$ Umfahren
		\item Dynamische Objekte $\rightarrow$ Bremsen
		\item Aspekte für Fitness definieren: Zeit ohne Kollision, passierte Teilstrecken
		\item Vorerst nur ein Fahrzeugtyp
		\item Wie werden Strecken gespeichert? XML? JSON?
	\end{itemize}
\end{itemize}
	\pagebreak
\section[Treffen mit Auftraggeber - 2017-10-24]{Treffen mit Auftraggeber}

\begin{enumerate}
	\item Datum: 2017-10-24
	\begin{enumerate}[label*=\arabic*.]
		\item Beginn: 15:05 Uhr
		\item Ende: 15:35 Uhr
	\end{enumerate}
	
	\item Teilnehmer:
	\subitem Jan~Amann, Khalid~Bellouch, Marvin~Janosch, Dennis~Szczepanski, Jan~Philip~Wahle
	
	\item Protokollant: Dennis~Szczepanski
\end{enumerate}
\ \\

\begin{itemize}
	\item Erster Einblick in Lasten- und Pflichtenheft sowie Klärung von Fragen zu:
	\begin{itemize}
		\item Einsatzgebiet der Software
		\item Produktübersicht
	\end{itemize}
	
	\item Besprechung der Gedanken aus Gruppentreffen
	\begin{itemize}
		\item Raster als Strecke $\rightarrow$ Hindernisse sollten als Fließkommawert gespeichert werden, nicht als weiteres Raster
		\item Streckenspeicherung $\rightarrow$ JSON praktisch
		\item Ladereihenfolge von Objekten
		\item Beispielhafte Produktleistungen $\rightarrow$ 64x64 Map, mit 100 Objekten
	\end{itemize}
\end{itemize}
	\pagebreak
\section[Treffen mit Auftraggeber - 2017-11-03]{Treffen mit Auftraggeber}

\begin{enumerate}
	\item Datum: 2017-11-03
	\begin{enumerate}[label*=\arabic*.]
		\item Beginn: 11:00 Uhr
		\item Ende: 11:30 Uhr
	\end{enumerate}
	
	\item Teilnehmer:
	\subitem Jan~Amann, Khalid~Bellouch, Dennis~Szczepanski
	
	\item Protokollant: Dennis~Szczepanski
\end{enumerate}
\ \\

\begin{itemize}
	\item Einsicht in Pflichtenheft 
	\item Einsicht in Klassendiagramm
	\begin{itemize}
		\item Polygone oder Splines als Begrenzung $\rightarrow$ Splines empfohlen, da diese das Arbeiten vereinfachen. Begrenzung als zwei mal stetig differenzierbare Funktion, bei einfachen Kurven nur testen, ob Fahrzeug sich zwischen den Funktionen befinden
		\begin{itemize}
			\item Problem $\rightarrow$ Kreuzungen
		\end{itemize}
		\item Editor und Simulation nutzen QWindow $\rightarrow$ QT als Paket assoziieren, und QWindow explizit nennen
	\end{itemize}
	\item Besprechen von kommender Abgabe
	\begin{itemize}
		\item Benutzerhandbuch
		\begin{itemize}
			\item Beschreibungen aller Menüs und Funktionen
		\end{itemize}
		\item Zusätzlich dynamische Diagramme abgeben
		\begin{itemize}
			\item Use-Case Diagramm
			\item Sequenzdiagramm
		\end{itemize}
	\end{itemize}
\end{itemize}
	
\end{document}